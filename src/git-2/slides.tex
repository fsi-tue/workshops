% Created 2019-06-17 Mon 17:57
% Intended LaTeX compiler: pdflatex
\documentclass[presentation, smaller]{beamer}
\usepackage[utf8]{inputenc}
\usepackage[T1]{fontenc}
\usepackage{graphicx}
\usepackage{grffile}
\usepackage{longtable}
\usepackage{wrapfig}
\usepackage{rotating}
\usepackage[normalem]{ulem}
\usepackage{amsmath}
\usepackage{textcomp}
\usepackage{amssymb}
\usepackage{capt-of}
\usepackage{hyperref}
\usetheme{CambridgeUS}
\usefonttheme{structurebold}
\author{Florian Engel}
\date{\today}
\title{Git II}
\frenchspacing
\hypersetup{
 pdfauthor={Florian Engel},
 pdftitle={Git II},
 pdfkeywords={},
 pdfsubject={},
 pdfcreator={Emacs 26.1 (Org mode 9.1.14)}, 
 pdflang={English}}
\begin{document}

\maketitle

\section{History durchsuchen}
\label{sec:orgc740d9b}
\begin{frame}[fragile,label={sec:org15d1ee5}]{Wer hat was verändert}
 \begin{verbatim}
> git log --author="Florian Engel"
\end{verbatim}
Zeigt alle commits an deren Autor Florian Engel ist
\end{frame}
\begin{frame}[fragile,label={sec:org5e7b411}]{Nach bestimmten commit messages suchen}
 \begin{verbatim}
> git log --grep "monetdb"
\end{verbatim}
Zeigt alle commits an in deren message "monetdb" vorkommt
\end{frame}
\begin{frame}[fragile,label={sec:org8bdc85e}]{Was wurde geändert}
 \begin{verbatim}
> git log --patch
\end{verbatim}
Zeigt zusätzlich zum log die diffs der einzelnen commits an
\end{frame}
\begin{frame}[fragile,label={sec:org61ccedc}]{Den log einer Datei anzeigen}
 \begin{verbatim}
> git log <datei>
\end{verbatim}
\begin{itemize}
\item Zeigt den log von <datei> an
\item kannn mit --patch verbunden werden
\end{itemize}
\end{frame}
\begin{frame}[fragile,label={sec:orgbc19348}]{Dateien vergleichen mit diff}
 \begin{verbatim}
> git diff <commit>
> git diff <commit> <commit>
\end{verbatim}
Zeigt die Änderungen zwischen 2 commits an
\end{frame}
\begin{frame}[label={sec:orgbf3491e}]{}
\begin{center}
Demo
\end{center}
\end{frame}
\section{Git commits korrieren}
\label{sec:org7b0ff0e}
\begin{frame}[fragile,label={sec:org48835f5}]{Änderen von letztem commit mit amend}
 \begin{verbatim}
> git commit --amend
\end{verbatim}
Bearbeitet den letztem commit
\end{frame}
\begin{frame}[fragile,label={sec:org9d722d7}]{Beispiel}
 \begin{verbatim}
> echo "The Room: 5 stars" > movie_rating.txt
> git add movie_rating.txt
> git commit -m "Added the room to rating"
> git commit --amend -m "Add the room to rating"
> # edit rating to 1 star
> git add movie_rating.txt
> git commit --amend --no-edit
\end{verbatim}
\end{frame}
\begin{frame}[label={sec:orga9be044}]{Ändern von mehrenen commits}
> git rebase --interactive HEAD\textasciitilde{}<Anzahl commits>
\begin{block}{Optionen:}
\begin{itemize}
\item p (pick): der commit soll so bleiben
\item r (reword): Die commit-message soll verändert werden
\item e (edit): amend auf diesen commit
\item s (squash): Der commit und der davor sollen zu einem zusammengeführt werden
\item f (fixup): Wie s aber ohne die commitmessage
\item d (drop): commit soll entfernt werden
\end{itemize}
\end{block}
\end{frame}
\section{Revision selection}
\label{sec:org0d2ee4d}
\begin{frame}[fragile,label={sec:orgbe9ed73}]{Wie funktioniert "\^{}"}
 \begin{verbatim}
> git log HEAD^
> git log HEAD^^
> git log HEAD^<n>
\end{verbatim}
\begin{itemize}
\item Gibt den ersten, zweiten,\ldots{} Parentcommit aus
\item `HEAD\^{}n' gibt den Parentcommit vom Parentcommit aus
\end{itemize}
\end{frame}
\begin{frame}[fragile,label={sec:orga7b5fed}]{Wie funktioniert "\textasciitilde{}"}
 \begin{verbatim}
> git log HEAD~
> git log HEAD~~
> git log HEAD~<n>
\end{verbatim}
\begin{itemize}
\item Gibt den ersten, zweiten,\ldots{} Parentcommit aus
\item `HEAD\textasciitilde{}n' gibt den 2. Parentcommit aus
\end{itemize}
\end{frame}
\begin{frame}[fragile,label={sec:org5314215}]{Was ist der unterschie zwischen "\textasciitilde{}" und "\^{}"}
 \begin{center}
\includegraphics[width=.9\linewidth]{./doubleDot.png}
\end{center}
\begin{itemize}
\item Welchen commit gibt \texttt{git show HEAD\textasciitilde{}2}
\item Welchen commit gibt \texttt{git show HEAD\textasciicircum{}2}
\end{itemize}
\end{frame}
\begin{frame}[label={sec:org22fca15}]{Wie sehe ich alle commits von topic die noch nicht in master sind}
>  git log master..topic
\end{frame}
\begin{frame}[label={sec:orgae633bb}]{Wie sehe ich alle commits die noch nicht in master sind}
> git log topic \^{}master
\end{frame}
\section{Merge-Strategien}
\label{sec:org88178dc}
\begin{frame}[fragile,label={sec:org16d3ac9}]{xours}
 \begin{verbatim}
> git merge -Xours <branch>
\end{verbatim}
Wenn ein Konflikt auftritt benutze den aktuellen branch
\end{frame}
\begin{frame}[fragile,label={sec:orgc5d6a56}]{xtheirs}
 \begin{verbatim}
> git merge -Xtheirs <branch>
\end{verbatim}
Wenn ein Konflikt auftritt benutze den anderen branch
\end{frame}
\begin{frame}[fragile,label={sec:orgeb4559f}]{ours}
 \begin{verbatim}
> git merge -s ours <branch>
\end{verbatim}
Nimm nur den aktuellen Branch
\end{frame}
\begin{frame}[fragile,label={sec:org05a193d}]{theirs}
 \begin{verbatim}
> git merge -s theirs <branch>
\end{verbatim}
\end{frame}
\section{Merge Konflikt auflösen}
\label{sec:orgb1f2c5b}
\begin{frame}[label={sec:org9a6f046}]{Merge Konflikt auflösen}
\begin{center}
Demo
\end{center}
\end{frame}
\section{Workflow}
\label{sec:orge2abcf8}
\begin{frame}[label={sec:org54e1806}]{Workflow}
\begin{center}
\includegraphics[width=.9\linewidth]{./workflow.png}
\end{center}
\begin{itemize}
\item Bei projekten kommen bestimmte branches häufig vor
\item master,stable,dev und feature branches
\end{itemize}
\end{frame}
\begin{frame}[label={sec:org3bb1150}]{master}
\begin{center}
\includegraphics[width=.9\linewidth]{./workflow.png}
\end{center}
\begin{itemize}
\item Das aktuell laufende Projekt
\item Enthält commits die an den Endnutzer weitergegeben werden können
\end{itemize}
\end{frame}
\begin{frame}[label={sec:orga5fc929}]{stable}
\begin{itemize}
\item Alle gut getesteten commits
\item Das Projekt in diesem Branch sollte wenig Bugs haben
\end{itemize}
\end{frame}
\begin{frame}[label={sec:orge20f883}]{dev}
\begin{center}
\includegraphics[width=.9\linewidth]{./workflow.png}
\end{center}
\begin{itemize}
\item Der aktuelle stand des Projekts
\item Entwickler wollen diesen noch nicht an Endnutzer weiter geben
\item Enthält eventuell noch viele Bugs
\end{itemize}
\end{frame}
\begin{frame}[label={sec:org3bb214a}]{feature branches}
\begin{center}
\includegraphics[width=.9\linewidth]{./workflow.png}
\end{center}
\begin{itemize}
\item Für jedes Feature einen branch
\item Wenn das Feature fertig ist, wird es in dev gemerged
\end{itemize}
\end{frame}
\end{document}