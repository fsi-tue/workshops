%% Linksammlung

% Math-mode Cheat Sheet
\newcommand{\heinkenCheat}{https://www.caam.rice.edu/\~heinken/latex/symbols.pdf}
\newcommand{\fontcalatogue}{https://tug.org/FontCatalogue/}
\newcommand{\ctanFontAwesome}{https://www.ctan.org/pkg/fontawesome5}

% Feedbacklink
\newcommand{\linkMaterial}{https://phictional.de/tutor/LaTeX/}
\newcommand{\linkFeedback}{https://phictional.de/tutor/feedback}



% Hello command for Section 'Commands'
\newcommand{\hello}[1][Phi]{Hallo #1, wie geht's so?}

% Environments
\newenvironment{para}[1]{
    \begin{minipage}{1.5em}
        \rotatebox{90}{\textsc{#1}}
    \end{minipage}\begin{minipage}{\linewidth}
}{
    \end{minipage}\smallskip
}

\newcounter{joke}
\newenvironment{joke}{
    \refstepcounter{joke}
    \noindent\colorbox{gray!50!white}{
        \textbf{Witz~\thejoke}
    } \\[.5em]
}{\medskip}


%% Aufgaben
% Aufgabe 1
\newcommand{\bbR}{\mathbb{R}}
\newcommand{\bbZ}{\mathbb{Z}}
\newcommand{\bbQ}{\mathbb{Q}}
\newcommand{\bbN}{\mathbb{N}}
\newcommand{\bbC}{\mathbb{C}}

% Aufgabe 2
\newcommand{\aufgabe}[2]{
    \section*{Aufgabe #1\hfill\small\textcolor{gray}{(#2 Punkte)}}
}

% Aufgabe 3
\newenvironment{loesung}[1][Lösung]{
    \medskip
    \noindent\textbf{#1}: 
    
    \medskip\noindent
}{\\\smallskip}


%% FONTS
% https://tug.org/FontCatalogue/comicneue/
\newcommand{\comicneue}{\fontfamily{ComicNeue-TLF}\selectfont}
\DeclareTextFontCommand{\textcn}{\comicneue}



