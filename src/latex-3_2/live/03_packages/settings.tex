%==================================================================
% 		Preface in its own file
%==================================================================

\usepackage[utf8]{inputenc}
\usepackage[ngerman]{babel}
\usepackage{amsmath}
\usepackage{amssymb}
\usepackage{fancyhdr}
\usepackage{color}
\usepackage{graphicx}
\usepackage{lastpage}
\usepackage{listings}
\usepackage{tikz}
\usepackage{pdflscape}
\usepackage{subfigure}
\usepackage{float}
\usepackage{polynom}
\usepackage{hyperref}
\usepackage{tabularx}
\usepackage{forloop}
\usepackage{geometry}
\usepackage{listings}
\usepackage{fancybox}
\usepackage{tikz}
\usepackage{algpseudocode,algorithm,algorithmicx}

%------------------------------------------------------------------------------
%     START: Our Commands & Environments
%------------------------------------------------------------------------------
\usepackage{lipsum} % Für Fülltext

%% Commands
% Slide 7
\newcommand{\hello}[1][Phi]{Hallo #1, wie geht's so?}

% Slide 9
%\renewcommand{\square}{^2}


%% Environments
% Slide 12
\newenvironment{para}[1]{
	\begin{minipage}{1.5em}
		\rotatebox{90}{\textsc{#1}}
	\end{minipage}\begin{minipage}{\linewidth}
	}{\end{minipage}\smallskip}

% Slide 13
\newcounter{joke}
\newenvironment{joke}{
	\refstepcounter{joke}
	\noindent
	\colorbox{gray!50!white}{
		\textbf{Witz~\thejoke}
	} \\[.5em]
}{\medskip}


%% Aufgaben
% Aufgabe 1: Zahlenräume
\newcommand{\bbR}{\mathbb{R}}
\newcommand{\bbZ}{\mathbb{Z}}
\newcommand{\bbQ}{\mathbb{Q}}
\newcommand{\bbN}{\mathbb{N}}
\newcommand{\bbC}{\mathbb{C}}

% Aufgabe 2: Aufgaben-Section mit Punkten
\newcommand{\aufgabe}[2]{
	\section*{Aufgabe #1\hfill\small\textcolor{gray}{(#2 Punkte)}}
}

% Aufgabe 3: Lösungs-Umgebung
\newenvironment{loesung}[1][Lösung]{
	\medskip
	\noindent\textbf{#1}: 
	
	\medskip\noindent
}{\\\smallskip}


%% Fonts

% Andere "Schriftarten"
% https://tug.org/FontCatalogue/
\usepackage[T1]{fontenc} % Notwendig für so ziemlich jede Schriftart
% \usepackage{fontspec}  % Für Xe(La)TeX und Lua(La)TeX

% Dokumentenweit
% https://tug.org/FontCatalogue/comicneue/
%\usepackage[default]{comicneue}

% Inline
\newcommand{\comicneue}{\fontfamily{ComicNeue-TLF}\selectfont}
\DeclareTextFontCommand{\textcn}{\comicneue}


%% Aufgaben
% Aufgabe 4: Neue Monospaced Schriftart
% https://tug.org/FontCatalogue/firamono/
\usepackage{FiraMono}

% Aufgabe 5: Awesome Fonts
% https://www.ctan.org/pkg/fontawesome5
% https://ftp.fau.de/ctan/fonts/fontawesome5/doc/fontawesome5.pdf
\usepackage{fontawesome5}



%------------------------------------------------------------------------------
%     END: Our Commands & Environments
%------------------------------------------------------------------------------

%Definiere Let-Command für algorithmen
\newcommand*\Let[2]{\State #1 $\gets$ #2}

%Größe der Ränder setzen
\geometry{a4paper,left=3cm, right=3cm, top=3cm, bottom=3cm}

%Kopf- und Fußzeile
\pagestyle {fancy}
\fancyhead[L]{Tutor: \TUTOR}
\fancyhead[C]{\COURSE}
\fancyhead[R]{\today}

\fancyfoot[L]{}
\fancyfoot[C]{}
\fancyfoot[R]{Seite \thepage /\pageref*{LastPage}}

%Formatierung der Überschrift, hier nichts ändern
\def\header#1#2{
	\begin{center}
		{\Large Übungsblatt #1}\\
		{(Abgabetermin #2)}
	\end{center}
}

%Definition der Punktetabelle, hier nichts ändern
\newcounter{punktelistectr}
\newcounter{punkte}
\newcommand{\punkteliste}[2]{%
	\setcounter{punkte}{#2}%
	\addtocounter{punkte}{-#1}%
	\stepcounter{punkte}%<-- also punkte = m-n+1 = Anzahl Spalten[1]
	\begin{center}%
		\begin{tabularx}{\linewidth}[]{@{}*{\thepunkte}{>{\centering\arraybackslash} X|}@{}>{\centering\arraybackslash}X}
			\forloop{punktelistectr}{#1}{\value{punktelistectr} < #2 } %
			{%
				\thepunktelistectr &
			}
			#2 &  $\Sigma$ \\
			\hline
			\forloop{punktelistectr}{#1}{\value{punktelistectr} < #2 } %
			{%
				&
			} &\\
			\forloop{punktelistectr}{#1}{\value{punktelistectr} < #2 } %
			{%
				&
			} &\\
		\end{tabularx}
	\end{center}
}
