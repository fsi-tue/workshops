% !TeX spellcheck = en_GB
%%%%%%%%%%%%%%%%%%%%%%%%%%%%%%%%%%%%%%%%%
% Beamer Presentation
% LaTeX Template
% Version 1.0 (10/11/12)
%
% This template has been downloaded from:
% http://www.LaTeXTemplates.com
%
% License:
% CC BY-NC-SA 3.0 (http://creativecommons.org/licenses/by-nc-sa/3.0/)
%
%%%%%%%%%%%%%%%%%%%%%%%%%%%%%%%%%%%%%%%%%

%----------------------------------------------------------------------------------------
%     PACKAGES AND THEMES
%----------------------------------------------------------------------------------------
% !TeX TXS-program:compile = txs:///pdflatex/[--shell-escape]
\documentclass{beamer}
%\usepackage[ngerman]{babel}


\mode<presentation> {

% The Beamer class comes with a number of default slide themes
% which change the colors and layouts of slides. Below this is a list
% of all the themes, uncomment each in turn to see what they look like.

%\usetheme{default}
%\usetheme{AnnArbor}
%\usetheme{Antibes}
%\usetheme{Bergen}
%\usetheme{Berkeley}
%\usetheme{Berlin}
%\usetheme{Boadilla}
%\usetheme{CambridgeUS}
%\usetheme{Copenhagen}
%\usetheme{Darmstadt}
%\usetheme{Dresden}
%\usetheme{Frankfurt}
%\usetheme{Goettingen}
%\usetheme{Hannover}
%\usetheme{Ilmenau}
%\usetheme{JuanLesPins}
%\usetheme{Luebeck}
\usetheme{Madrid}
%\usetheme{Malmoe}
%\usetheme{Marburg}
%\usetheme{Montpellier}
%\usetheme{PaloAlto}
%\usetheme{Pittsburgh}
%\usetheme{Rochester}
%\usetheme{Singapore}
%\usetheme{Szeged}
%\usetheme{Warsaw}

% As well as themes, the Beamer class has a number of color themes
% for any slide theme. Uncomment each of these in turn to see how it
% changes the colors of your current slide theme.

%\usecolortheme{albatross}
%\usecolortheme{beaver}
%\usecolortheme{beetle}
%\usecolortheme{crane}
%\usecolortheme{dolphin}
%\usecolortheme{dove}
%\usecolortheme{fly}
%\usecolortheme{lily}
%\usecolortheme{orchid}
%\usecolortheme{rose}
%\usecolortheme{seagull}
%\usecolortheme{seahorse}
%\usecolortheme{whale}
%\usecolortheme{wolverine}

%\setbeamertemplate{footline} % To remove the footer line in all slides uncomment this line
%\setbeamertemplate{footline}[page number] % To replace the footer line in all slides with a simple slide count uncomment this line

%\setbeamertemplate{navigation symbols}{} % To remove the navigation symbols from the bottom of all slides uncomment this line
}

\usepackage{graphicx} % Allows including images
\usepackage{booktabs} % Allows the use of \toprule, \midrule and
% \bottomrule in tables
\usepackage{listings}
\usepackage{parcolumns}
\usepackage[nocenter]{qtree}
\usepackage{minted}
\usepackage{eurosym}
\usepackage{qrcode}
\usepackage{xcolor}
\definecolor{LightGray}{gray}{0.95}
%----------------------------------------------------------------------------------------
%     TITLE PAGE
%----------------------------------------------------------------------------------------

\title[]{Introduction to programming in Python} % The short title appears at the bottom of every slide, the full title is only on the title page

\author{Jules Kreuer} % Your name
\institute[FSI] % Your institution as it will appear on the bottom of every slide, may be shorthand to save space
{
Uni Tübingen\\ % Your institution for the title page
\medskip
\textit{fsi@fsi.uni-tuebingen.de}\\
\textit{contact@juleskreuer.eu}\\
}
\date{\today} % Date, can be changed to a custom date

\begin{document}

\begin{frame}
\titlepage % Print the title page as the first slide
\end{frame}

%----------------------------------------------------------------------------------------
%     PRESENTATION SLIDES
%----------------------------------------------------------------------------------------

\begin{frame}
	Based on:\\\\
	Ana Bell, Eric Grimson, and John Guttag. \\
	6.0001 Introduction to Computer Science and Programming in Python.\\
	Fall 2016. Massachusetts Institute of Technology: MIT OpenCourseWare\\
	https://ocw.mit.edu.\\
	License: Creative Commons BY-NC-SA.\\\\
	
	Nick Parlante, John Cox, Steve Glassman, Piotr Kaminksi, Antoine Picard.\\
	Google's Python Class.\\
	July 2015. Google LLC\\
	License: Creative Commons BY 2.5.	
\end{frame}

\begin{frame}
	\begin{block}{}<1->
		\textbf{Part 1: Hello World}\\
		- Introduction\\ 
		- Installation \\
		- REPL\\
	\end{block}
	\begin{exampleblock}{}
	Break
	\end{exampleblock}
	\begin{block}{}<2->
	\textbf{Part 2: Basics}\\
	- Common operators\\
	- Data types, type-casting\\
	- Lists, dicts\\
	- Control flow: for, while, break, continue\\
	\end{block}
	\begin{exampleblock}{}<2->
		Break
	\end{exampleblock}
\end{frame}

\begin{frame}
	\begin{block}{}
		\textbf{Part 3: Abstraction}\\
		- Functions, variable scope, lambda\\
		- Files / IO\\
		- Exceptions\\
		- recursion\\
		- Objects, Classes\\
	\end{block}
	\begin{exampleblock}{}
	Break
	\end{exampleblock}
	\begin{block}{}<2->
		\textbf{Part 4: Development}\\
		- Arguments\\
		- Modules / imports\\
		- Virtual-Envs\\
		- Tests
	\end{block}
\end{frame}

\begin{frame}[fragile]
	\frametitle{Demo}
	\begin{minted}{bash}
jules@T480:~$ python3
Python 3.8.10 (default, Nov 26 2021, 20:14:08) 
[GCC 9.3.0] on linux
Type "help", "copyright", "credits" or "license"...
	\end{minted}
\begin{minted}{python}
>>> a = 5
>>> a
5
>>> a = "Hello World"
>>> a
'Hello World'
>>> a + "!"
'Hello World!'
>>> 
	\end{minted}
\end{frame}

\begin{frame}
	\frametitle{Installation / REPL}
	\begin{center}
		\url{https://www.python.org/downloads/}\\
		Debian / Ubuntu: \textit{sudo apt install python3}\\
	\end{center}
	\begin{center}
		Type in your shell: \textit{python3}
	\end{center}
\end{frame}

\begin{frame}
	\begin{figure}
		\includegraphics[width=12cm]{figures/console.png}
		\caption{Python3 REPL}
	\end{figure}
\end{frame}

\begin{frame}[fragile]
	\begin{block}{Running code}
		- REPL\\
		- python3 file args
	\end{block}
	\begin{example}
			\begin{minted}{bash}
python3 hello-world.py
		\end{minted}
	\end{example}
\end{frame}

\begin{frame}
	\frametitle{Combining Editor and Interpreter}
	\begin{figure}
		\includegraphics[width=8cm]{figures/vs-code.png}
		\caption{VS Codium}
	\end{figure}
\end{frame}
\begin{frame}
	\textbf{Possible IDEs / Editors:}\\
	- VS Codium: \url{https://vscodium.com/}\\
	- PyCharm: \url{https://www.jetbrains.com/pycharm/}\\
	- Atom: \url{https://atom.io/}\\
	- ...
\end{frame}

\begin{frame}[fragile]
	\frametitle{hello-world.py}
	- Content: \mintinline{python}{print("Hello World")}\\
	- Run it!
\end{frame}
%%%%%%%%%%%%%%%%%%%%%%%%%%%%%%%%%%%%%%%%%%%%%%%%%%%%%%%%%%%%%%%%%%%%%%%%%%%%%%%%%%%%%%%%%%%%%
% Part 2
\begin{frame}[fragile]
	\frametitle{Basic operators and types}
	Just like `any other' language.
	\begin{block}{Math}
		\begin{minted}{python}
s = (a + b - c) / d * e
p = a ** 2 # a to the power of 2
b = a^2    # bitwise shifting
m = a%2    # mod
		\end{minted}
	\end{block}
	\begin{exampleblock}{Numeric types}
	\begin{minted}{python}
int, float, complex
i = 1 = int("1") = int(1.0)
f = 4.2
c = 4+2j
	\end{minted}
\end{exampleblock}
\end{frame}
\begin{frame}[fragile]
	\begin{block}{Strings}
		\begin{minted}{python}
s = "Hello " + "World"
c = "A" * 10 + "HHHH"
S = s.upper()
length = len(S)   # Returns Integer
pos = s.find("W") # Return Integer (Position of first W)
		\end{minted}
	\end{block}
	\begin{exampleblock}{Text types}
		\begin{minted}{python}
str
s = str(1)
		\end{minted}
	\end{exampleblock}
\end{frame}

\begin{frame}[fragile]
	\begin{block}{Booleans}
		\begin{minted}{python}
a = (True or False) and not False
		\end{minted}
	\end{block}
	\begin{exampleblock}{Boolean types}
		\begin{minted}{python}
bool
t = bool(1) = bool("Not Empty")
f = bool(0) = bool("")
		\end{minted}
	\end{exampleblock}
\end{frame}

\begin{frame}[fragile]
	\begin{block}{Comparision}
		\begin{minted}{python}
<, >, ==, !=, <=, >=
		\end{minted}
	\end{block}
	\begin{example}{}
		\begin{minted}{python}
t = 3 < 5 
t = not "A" == "B"
f = 4.2 == 2
f = 0 == "Hello" # Comparision in between types is possible
		\end{minted}
	\end{example}
\end{frame}

\begin{frame}[fragile]
	\frametitle{01-types.py}
	\begin{exampleblock}{Exercise}
	Desired output: `The sum of 41.8 and 0.2 is 42'.\\
	Use following variables:
		\begin{minted}{python}
i = 41.8
f = 0.2
prefix = "The sum of "
		\end{minted}
	\end{exampleblock}
\end{frame}

%%%%%%%%%%%%%%%%%%%%%%%%%%%%%%%%%%%%%%%%%%%%%%%%%%%%%%%%%%%%%%%%%%%%%%%%%%%
% Lists / Dicts / Tuples
\begin{frame}[fragile]
	\begin{block}{Lists}<1->
Mutable, dynamic in length, non-homogenous, ordered
\begin{minted}{python}
aList = [1, 2, 3, 4, "What?", 6]
aList[0]             # -> 1
aList[4:]            # -> ['What?', 6]
aList[1::2]          # -> [2, 4, 6]  
aList[-1]            # -> 6
aList.append(7)      # -> [..., 6, 7]
aList.extend([8,9])  # -> [..., 6, 7, 8, 9]
aList[0] = "New Zero"
general form: [from:to:step/order]
\end{minted}
	\end{block}
	\begin{block}{Tuples}<2->
	Non-Mutable, fixed length, non-homogenous
	\begin{minted}{python}
aTuple = ("A", "a", 1)
a[0] 	# -> "A"
	\end{minted}
\end{block}
\end{frame}

\begin{frame}[fragile]
	
	\begin{block}{Dicts}
	Mutable, dynamic in size, non-homogenous, unordered\footnote{Somehow..}
	\begin{minted}{python}
d = {"key": "value", 1: 3}
d["key"]     # -> "value"
d["new"] = 2 # Insert new value to d
d.keys()     # -> ["key", 1, "new"]
d.values()   # -> ["value, 3, 2]
d.items()    # -> [("key", "value"), (1, 3), ("new", 2)]
	\end{minted}
	\end{block}
\textbf{See: } \url{https://docs.python.org/3/tutorial/datastructures.html}
\end{frame}
%%%%%%%%%%%%%%%%%%%%%%%%%%%%%%%%%%%%%%%%%%%%%%%%%%%%%%%%%%%%%%%%%%%%%%%%%%%%%%
% If and Loops

\begin{frame}[fragile]
	\frametitle{Control flow: if, for, while, break, continue}
	Regular control flow with if:
	\begin{minted}{python}
if condition:
  doThis()
elif cond2:
  doThat()
else:
  otherWise()
	\end{minted}
\end{frame}

\begin{frame}[fragile]
Looping has two different approaches:
\begin{block}{while / condition}<1->
	\begin{minted}{python}
i = 0
while i < 10:
  print(i)
  i = i + 1
	\end{minted}
\end{block} 
\begin{block}{for / iterable}<2->
	\begin{minted}{python}
for element in iterable:
  print(e)
	\end{minted}
\end{block} 
\end{frame}

\begin{frame}[fragile]
Iterables: something with an order and members.
\begin{example}
	\begin{minted}{python}
tuples = (0, 1,2,3,4)
lists  = [0, 1,2,3,4]
string = "Hello World"
dicts = {"a", 2}  
range(0, 6, 2) # start, stop, interval
               # somehow comparable to [0,2,4]
file objects
...
	\end{minted}
\end{example}
\end{frame}

\begin{frame}[fragile]
	\begin{block}{for / iterable}<1->
		\begin{minted}{python}
for element in iterable:
  print(e)
		\end{minted}
	\end{block}
\begin{example}<2->
		\begin{minted}{python}
for i in range(5):
  print(e) # 0, 1, 2, 3, 4
for c in "Hello World":
  print(e) # Every char
for k in {"k": "v", "k2": "v2"}:
  print(k) # Only the keys
for k, v in {"k": "v", "k2": "v2"}.items():
  print(k, v) # Unpacking
\end{minted}
\end{example}
\end{frame}

\begin{frame}[fragile]
\textbf{Unpacking}:\\
- Object with ordered members\\
- Number of vars equal to members\footnote{\mintinline{python}{x, *xs  = [1, 2, 3, 4]}  $\rightarrow$ \mintinline{python}{xs = [2,3,4]}}.
\begin{example}
	\begin{minted}{python}
a, b, c = (1, 2, 3)
a, b,   = [1,2]
	\end{minted}
\end{example}
\end{frame}

\begin{frame}[fragile]
	Exit the loop early?
	\begin{block}{Break}
	\begin{minted}{python}
while True:   # works for "for i in .." aswell 
  doThis()
  if exitCondition:
     break
	\end{minted}
	\end{block}
	Skip to the next element?
	\begin{block}{Continue}
		\begin{minted}{python}
for i in range(4):
  if i == 2:
    continue
  print(i)
-> 0, 1, 3
		\end{minted}
	\end{block}
\end{frame}


\begin{frame}[fragile]
	\frametitle{02-number-guess.py}
	\begin{exampleblock}{Exercise}
Implement a basic python number guessing game.\\
1. Generate a random number.\\
2. Ask for a guess.\\
3. Check if guess was correct.\\
4. If not, say if number was smaller / larger\\
5. Repeat from step 2, but only 7 times max.\\
~\\
Use following functions:
		\begin{minted}{python}
from random import randint(a, b)
randint(0,1024)  # random integer N such that a <= N <= b
input("Number?") # Takes input from user
		\end{minted}
	\end{exampleblock}
\end{frame}
% End of Part 2
%%%%%%%%%%%%%%%%%%%%%%%%%%%%%%%%%%%%%%%%%%%%%%%%%%%%%%%%%%%%%%%%%%%%%%%%%%%%%%%%%%%%%%%%%%%%%

% Part 3
\end{document} 